%TODO falta terminar, te lo pusheo para que veas que fuimos haciendo

En esta sección vamos a discutir acerca de los resultados expuestos en la sección de Resultados.

% Observamos que en los gráficos donde la probabilidad de pérdida de paquetes es 0, la cantidad de re-transmisiones es similar para distintas corridas. Esto se debe a que al no haber \emph{dropeos}, se re-envía solamente por \emph{timeout}.
% 
% Análogamente, cuando \emph{delay} y la varianza son altos, la cantidad de re-transmisiones en las distintas corridas no son similares sino que hay mucha variabilidad.

Con respecto a $\alpha$, observamos que cuando es 0 la aproximación del \emph{Sampled RTT} respecto al RTO suele ser considerablemente peor respecto del resto de las corridas. Esto lo evidenciamos, por ejemplo, en la Figura \ref{fig:alpha-var2-drop0}.

Esto no ese cumple, por ejemplo, cuando tenemos varianza, \emph{delay} y una probabilidad de pérdida de paquetes alto (como podemos ver en la Figura \ref{fig:alpha-var5-drop50-alto}). En estos casos, la relación del \emph{Sampled RTT} respecto del RTO no es buena para ningún $\alpha$, y en particular, las estimaciones del RTT son más parecidas que en los casos anteriores. 

Estos últimos casos los relacionamos con redes reales no estables. Mientras que casos donde la varianza es baja, y la probabilidad de pérdida de paquetes es baja, las relacionamos con redes reales estables.

Observamos que, en general, la probabilidad de pérdida de paquetes suele ser un factor determinante al momento de estimar el RTT. Podemos ver que en gráficos donde dicha probabilidad es baja, los distintos $\alpha$ usados aproximan bastante bien al RTT (a excepción de $\alpha$ = 0). Esto lo podemos ver en la Figura \ref{fig:alpha-var2-drop0}.

Análogamente, en las Figuras \ref{fig:alpha-var2-drop25} y \ref{fig:alpha-var2-drop50} podemos observar que a medida que crece la probabilidad de pérdida de paquetes, aumenta la volatilidad de los RTOs para todos los $\alpha$.

Sin embargo, cabe aclarar que otro factor importante es la varianza. Aún más, creemos que la varianza es un factor más importante que el \emph{delay} al momento de estimar el RTT. En las Figuras \ref{fig:alpha-var2-drop25-alto} (Varianza baja, \emph{delay} alto) y \ref{fig:alpha-var5-drop25} (Varianza alta, \emph{delay} bajo), podemos evidenciar justamente esto.

En la Figura \ref{fig:alpha-var2-drop25-alto}, como tenemos poca varianza, por más que el \emph{delay} sea alto, podemos estimar relativamente bien el RTT (con la excepción de $\alpha$ = 0). En contraposición, en la Figura \ref{fig:alpha-var5-drop25}, como la varianza es alta, obtenemos  muchos picos en el cálculo de la estimación del RTT. En dichas figuras, podemos observar que nos cuesta más estimar el RTT, teniendo varianza alta y delay bajo que en el caso opuesto.

En esta sección, estuvimos excluyendo a $\alpha$ = 0 y vamos a explicar a continuación por qué lo hicimos. Cuando $\alpha$ es 0, según el Algoritmo de Karn, el SRTT se mantiene fijo a lo largo de las distintas iteraciones de estimación de RTT. Recordamos, que el valor inicial de SRTT se define en la primer muestra como RTT. Esto implica, que en cada iteración, el SRTT no se modifica por más que el RTT varíe mucho.

Observamos que, cuando la varianza es baja por lo general los valores de $\alpha$ 0,125 y 0,2 se comportan lo suficientemente bien. Creemos que esto sucede, ya que cuando la varianza es baja es posible estimar el RTT debido a que justamente, las estimaciones parciales de RTT son similares al RTT real. Estos valores pertenecen a los valores recomendados en el Peterson\footnote{Computer Networks: A Systems Approach, Larry L. Peterson and Bruce S. Davie. Tercera edición.}.

Sin embargo, cabe aclarar que cuando tenemos varianza alta estimar el RTT se vuelve difícil ya que en cada muestra, el RTT estimado puede ser muy distinto a la anterior. Por consiguiente no hay ningún valor de $\alpha$ lo suficientemente bueno.

%TODO falta terminar beta

A partir de ahora, vamos a analizar la estimación del RTT respecto de $\beta$.

Observamos que, en general, la probabilidad de pérdida de paquetes suele ser un factor determinante al momento de estimar el RTT. Podemos ver que en gráficos donde dicha probabilidad es baja, los distintos $\alpha$ usados aproximan bastante bien al RTT. Esto lo podemos ver en la Figura \ref{fig:alpha-var2-drop0}.
